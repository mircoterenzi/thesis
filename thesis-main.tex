\documentclass[12pt,a4paper,openright,twoside]{book}
\usepackage[utf8]{inputenc}
\usepackage{disi-thesis}
\usepackage{code-lstlistings}
\usepackage{notes}
\usepackage{shortcuts}
\usepackage{acronym}

\school{\unibo}
\programme{Corso di Laurea in Ingegneria e Scienze Informatiche}
\title{Utilizzo di Neverlang per la creazione di Domain Specific Languages} %TODO: discuss the title with profs
\author{Terenzi Mirco}
\date{\today}
\subject{Programmazione ad Oggetti}
\supervisor{Prof. Viroli Mirko}
\cosupervisor{Prof. Aguzzi Gianluca}
\session{II}
\academicyear{2023-2024}

% Definition of acronyms
\acrodef{FOP}{Feature Oriented Programming}
\acrodef{DSL}{Domain Specific Language}
\acrodef{AST}{Abstract Syntax Tree}


\mainlinespacing{1.241} % line spacing in mainmatter, comment to default (1)

\begin{document}

\frontmatter\frontispiece

\begin{abstract}    
Max 2000 characters, strict.
\end{abstract}

%----------------------------------------------------------------------------------------
\tableofcontents   
%\listoffigures     % (optional) comment if empty
%\lstlistoflistings % (optional) comment if empty
%----------------------------------------------------------------------------------------

\mainmatter

%----------------------------------------------------------------------------------------
\chapter{Introduzione}
\label{chap:introduzione}
%----------------------------------------------------------------------------------------

Write your intro here.

\paragraph{Struttura della Tesi}

%----------------------------------------------------------------------------------------
\chapter{Background}
\label{chap:background}
%----------------------------------------------------------------------------------------


%----------------------------------------------------------------------------------------
\chapter{Requisiti}
\label{chap:requisiti}
%----------------------------------------------------------------------------------------


%----------------------------------------------------------------------------------------
\chapter{Design e Implementazione}
\label{chap:Imple}
%----------------------------------------------------------------------------------------


%----------------------------------------------------------------------------------------
\chapter{Validazione e Conclusioni}
\label{chap:conclusioni}
%----------------------------------------------------------------------------------------


%----------------------------------------------------------------------------------------
% BIBLIOGRAPHY
%----------------------------------------------------------------------------------------

\backmatter

\nocite{*} % Remove this as soon as you have the first citation

\bibliographystyle{alpha}
\bibliography{bibliography}

\begin{acknowledgements} % this is optional
Optional. Max 1 page.
\end{acknowledgements}

\end{document}
