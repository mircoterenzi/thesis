\documentclass[12pt,a4paper,openright,twoside]{book}
\usepackage[utf8]{inputenc}
\usepackage{disi-thesis}
\usepackage{code-lstlistings}
\usepackage{notes}
\usepackage{shortcuts}
\usepackage{acronym}

\school{\unibo}
\programme{Corso di Laurea in Ingegneria e Scienze Informatiche}
\title{Utilizzo di Neverlang per la creazione di Domain Specific Languages} %TODO: discuss the title with profs
\author{Terenzi Mirco}
\date{\today}
\subject{Programmazione ad Oggetti}
\supervisor{Prof. Viroli Mirko}
\cosupervisor{Prof. Aguzzi Gianluca}
\session{II}
\academicyear{2023-2024}

% Definition of acronyms
\acrodef{FOP}{Feature Oriented Programming}
\acrodef{DSL}{Domain Specific Language}
\acrodef{JVM}{Java Virtual Machine}
\acrodef{AST}{Abstract Syntax Tree}


\mainlinespacing{1.241} % line spacing in mainmatter, comment to default (1)

\begin{document}

\frontmatter\frontispiece

\begin{abstract}    
Max 2000 characters, strict.
\end{abstract}

%----------------------------------------------------------------------------------------
\tableofcontents   
%\listoffigures     % (optional) comment if empty
%\lstlistoflistings % (optional) comment if empty
%----------------------------------------------------------------------------------------

\mainmatter

%----------------------------------------------------------------------------------------
\chapter{Introduzione}
\label{chap:introduzione}
%----------------------------------------------------------------------------------------

Write your intro here.

\paragraph{Struttura della Tesi}

%----------------------------------------------------------------------------------------
\chapter{Background}
\label{chap:background}
%----------------------------------------------------------------------------------------

\section{Domain-Specific Languages}
In modo del tutto opposto rispetto ai linguaggi general-purpose, progettati per poter essere utilizzati in ogni contesto con un’efficienza e 
un grado d’espressività relativamente uguali, i \ac{DSL} sono ottimizzati per uno specifico ambito e risultano essere, in molti casi, una soluzione 
molto più naturale rispetto a quella fornita dai primi \cite{Hudak1997}. Tra gli esempi più comuni di \ac{DSL} troviamo SQL, LaTeX (utilizzato anche per 
la scrittura di questo documento) e CSS.

Nonostante la definizione di \ac{DSL} sia chiara, non è altrettanto immediato definire se un linguaggio sia o meno un \ac{DSL}. In questo caso, ci sono 
quattro fattori chiave da osservare \cite{Fowler2010}:
\begin{itemize}
    \item Un \ac{DSL} è un linguaggio di programmazione pensato per eseguire delle operazioni su un computer. Per tale motivo, la sua struttura 
    dovrebbe essere progettata in modo da essere facile da comprendere per gli esseri umani e, al tempo stesso, eseguibile da un compilatore.
    \item Essendo un linguaggio, deve avere un senso di fluidità e la sua espressività deve essere derivata non solo da un’espressione 
    individuale, ma anche dall’unione di più istruzioni.
    \item Per essere coerente con la sua natura specializzata, un \ac{DSL} dovrebbe avere le caratteristiche minime per poter supportare il 
    dominio applicativo di interesse ed evitare funzioni non strettamente necessarie che potrebbero rendere il linguaggio più difficile 
    da usare e da comprendere.
\end{itemize}

%TODO: implement from draft


\section{Neverlang}
Neverlang Language Workbench è un framework sviluppato presso l’Università di Milano dal professor Cazzola e dai suoi collaboratori, il cui 
scopo è favorire lo sviluppo di linguaggi di programmazione, in particolare secondo il paradigma di programmazione feature-oriented.

È basato sull’idea che i linguaggi di programmazione abbiano un’intrinseca divisione modulare in più caratteristiche, o \textit{features}, ciascuna 
delle quali è implementata da un componente specifico. In accordo con tale visione, l’obiettivo del framework è definire i linguaggi tramite 
una divisione in frammenti, chiamati moduli, ognuno dei quali si occupa di implementare una specifica caratteristica e, infine, tramite la 
combinazione dei diversi moduli, ottenere un linguaggio di programmazione specifico per il contesto applicativo richiesto, ossia un  
\ac{DSL} \cite{NeverlangWebsite}.

In particolare, all’interno di ogni modulo vengono definite due parti principali:
\begin{itemize}
    \item la \textbf{sintassi}, utilizzando una grammatica formale;
    \item la \textbf{semantica}, in funzione della sintassi e sfruttando i vari elementi non-terminali e i loro attributi. Inoltre, il 
    comportamento del componente può essere suddiviso in diverse fasi, ciascuna identificata da un ruolo specifico del componente.
\end{itemize}
Successivamente, i componenti del linguaggio vengono definiti combinando definizioni di sintassi e semantica provenienti da diversi moduli, 
all’interno di elementi detti \textit{slice} \cite{Vacchi2015}.

Tra i vantaggi principali di Neverlang troviamo \cite{Cazzola2012}:
\begin{itemize}
    \item \textbf{Modularità}: Ognuno dei moduli che compongono il linguaggio viene compilato separatamente, permettendo di utilizzarne uno o 
    più di uno (in tal caso aggregandoli in uno \textit{slice}) all’interno di altri linguaggi.
    \item \textbf{Riutilizzo}: Neverlang offre la possibilità di riutilizzare frammenti di linguaggio in più di un contesto. Ad esempio, un 
    frammento può utilizzare la sintassi di un altro frammento definito in precedenza e ridefinire la semantica, o viceversa. Inoltre, è 
    possibile ridefinire l’ordine dei simboli non-terminali utilizzati nella sintassi o nella semantica importata.
    \item \textbf{Estensibilità}: L’architettura modulare utilizzata all’interno di Neverlang facilita l’estensione di linguaggi esistenti. 
    Per aggiungere nuove funzionalità non è necessario modificare il codice, ma è sufficiente integrare un nuovo \textit{slice}.
\end{itemize}


\section{Java}
In aggiunta a Neverlang, per la realizzazione del progetto è stato utilizzato il linguaggio di programmazione Java. Java è un linguaggio di 
programmazione ad alto livello, orientato agli oggetti e a tipizzazione statica, sviluppato da Sun Microsystems nel 1991. È molto diffuso e 
ben supportato, con una vasta comunità di sviluppatori e una grande quantità di librerie. Uno degli obiettivi principali di Java è quello di 
essere il più possibile indipendente dalla piattaforma di esecuzione, permettendo di scrivere una volta il codice e farlo eseguire su qualsiasi 
\ac{JVM}, indipendentemente dall'architettura del computer \cite{IBMWebsite}.

Java è stato utilizzato per la realizzazione del progetto in quanto Neverlang è sviluppato per essere completamente integrato con esso.
Il suo compilatore, nlgc, è stato sviluppato per poter convertire il codice scritto utilizzando il DSL di Neverlang, generando un nuovo codice
supportato dalla \ac{JVM}. Inoltre, Neverlang permette di utilizzare Java (ma non solo; anche Scala, ad esempio, è supportato) come linguaggio 
per la definizione della semantica all'interno dei moduli del \ac{DSL}. Ciò è possibile in quanto gli accessi a variabili non-terminali, 
definiti all'interno della sintassi, sono sostituiti dallo specifico plug-in con accessi alla reale rappresentazione interna del linguaggio. 
In particolare, l'accesso alle variabili viene effettuato tramite una chiamata all'n-esimo figlio dell'\ac{AST} \cite{Cazzola2013}.

%----------------------------------------------------------------------------------------
%\chapter{Requisiti}
%\label{chap:requisiti}
%----------------------------------------------------------------------------------------


%----------------------------------------------------------------------------------------
%\chapter{Design e Implementazione}
%\label{chap:Imple}
%----------------------------------------------------------------------------------------


%----------------------------------------------------------------------------------------
%\chapter{Validazione e Conclusioni}
%\label{chap:conclusioni}
%----------------------------------------------------------------------------------------


%----------------------------------------------------------------------------------------
% BIBLIOGRAPHY
%----------------------------------------------------------------------------------------

\backmatter

\nocite{*} % Remove this as soon as you have the first citation

\bibliographystyle{alpha}
\bibliography{bibliography}

%\begin{acknowledgements} % this is optional
%Optional. Max 1 page.
%\end{acknowledgements}

\end{document}
